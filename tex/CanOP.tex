\section{CanOp}
\label{sec_CanOp}
CanOp est le nom qui a été donné au projet de crée une sonde nouvelle génération pour la mine Somaïr au Niger. Cette sonde est composée de 3~pièces. 
\begin{itemize}
    \item 2 Sondes de rayonnement Gamma fournissent par la société Geovista
    \item une partie électronique qui inclue une batterie.
    \item Un GPS différentiel fourni par Ophelia 
\end{itemize}
Un opérateur utilise cette sonde en connexion avec une tablette pour déterminer ou extraire du minerai. 
\subsection{Les sondes Gamma}
\label{ssec_sonde}
Les sondes gamma de cet appareil proviennent de chez Ophelia et sont composées de deux parties.
\begin{figure}
    \centering
    \includegraphics[width=0.7\textwidth]{img/she/Photomultiplier_coupled_to_a_scintillator_-_fr.png}
    \caption[Shema d'une sonde gamma NaI]{Schéma d'une sonde gamma NaI. Source~: \href{https://commons.wikimedia.org/wiki/File:Photomultiplier_coupled_to_a_scintillator_-_fr.png}{Qwerty123uiop}, \href{https://creativecommons.org/licenses/by-sa/3.0}{CC BY-SA~3.0}, via Wikimedia Commons}
    \label{fig_detecteur_gamma}
\end{figure}
\begin{description}
    \item[Un crystal NaI] Ce cristal a la propriété d'absorber les photons haut énergie des rayons gamma pour les réémettre comme des photons plus basse énergie (voir partie gauche de la \cref{fig_detecteur_gamma})~\cite{site:explication_NaI}
    \item[Un tube photomultiplicateur] ce tube permet de convertir un photon en un photoélectron qui est ensuite multiplié par le tube pour être converti en signaux électriques. (Voir partie droite de la \cref{fig_detecteur_gamma})~\cite{site:explication_NaI}
\end{description}
Pour diverse raison, il y a deux sondes dans la partie basse de la CanOp. L'opérateur peut choisir avec quelle sonde il souhaite effectuer la mesure (bien que les valeurs des deux sondes sont enregistrées dans la base de données).

\subsection{Le GPS différentiel}
\label{ssec_Gps_differenciel}
Pour que la CanOp puisse fonctionner correctement, il faut qu'elle soit situer très précisément ($\pm$ 10~cm sur les axes x et y et $\pm$ 1~cm sur les axes z) or un GPS classique n'arrive qu’a atteindre $\pm$~3~m horizontalement et $\pm$ 5~m verticalement dus notamment aux perturbations atmosphériques que subisse les signaux. 
\begin{figure}
    \centering
    \includegraphics{img/she/GPS-mode-Naturel-5-10m.png}
    \caption[Source d'erreur des GPS]{Schéma présentant les sources d'erreur des GPS. Source~: Orphéon}
    \label{fig_GPS_error_source}
\end{figure}

Une des solutions possibles pour contourner ces problèmes est d'utiliser un GPS différentiel. Le principe de fonctionnement est simple, une station fixe à proximité de notre zone de mesure reçoit également les signaux GPS et en connaissant sa position précise peut calculer et transmettre les corrections nécessaires. \cite{site:GPS_diff}
\begin{figure}
    \centering
    \includegraphics[width=0.5\textwidth]{img/she/Real_time_kinematic.png}
    \caption[Shema d'un systeme GPS differenciel]{Schéma d'un système GPS différentiel. Source~: \href{https://commons.wikimedia.org/wiki/File:Real_time_kinematic.svg}{TS Eriksson}, \href{https://creativecommons.org/licenses/by-sa/4.0}{CC BY-SA~4.0}, via Wikimedia Commons}
    \label{fig_RTK}
\end{figure}

\subsection{L'électronique}

    