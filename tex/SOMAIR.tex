\section{Les mine d'uranium d'Orano}
Orano possède plusieurs mines dans le monde avec différents manieurs d'extraction selon les conditions géologiques présentes et de la nature du gisement. C'est différent mine qui sont~:
\begin{itemize}
    \item Muyunkum et Tortkuduk au Kasakstan avec une recuperation in situ (voir \cref{ssec_insitu})
    \item McAthure River et Cigar Lake au Canada avec des methode de "jetboring" (voir \cref{ssec_raisebore,ssec_jetbore,ssec_boxbore})
    \item Somaïr au Niger avec une mine a ceil Ouvert
\end{itemize}

\subsection{Somaïr}
\subsubsection{L'exploration}
\par Avant le debut de l'extraction, des geologue ont realiser des etude pour trouver d'eventuel gisment. S'ils soupsone la presence d'uranium, les geologuevont realiser des campagne de sondage succesive \footnote{Soit une carotte ou un forage dans lequel on abaisser un 
sonde gamma. Cela permet d'etablire le gisment en 3D}. La maille de sondage seras affiner jusqu'a avoir des forage espacer de 25m. 
\subsubsection{L'extraction}
Si la décision de passer en production est prise alors on va venir enlever toute la roche au-dessus du gisement (50 a 70~m à Somaïr) et l’on va affiner le sondage jusqu'à une maille de 5~m*5~m qui va permettre de modéliser au mieux la distribution d'uranium dans le sol. Enfin, la fosse sera divisée en carrés de 2,5~m de large sur 2,5~m de longueur sur 0,5~m de profondeur que l'on appellera "slab" par la suite. Pour extraire ces slabs, on enterre juste assez d'explosif pour fragiliser la roche et permettre qu'une pelle mécanique puisse extraire la slab pour la charge dans un camion.
\subsubsection{Classification des slabs}
Pour savoir comment traiter ces slabs  après extraction, nous les catégorisons en 7~classes de M0 à  en fonction de leur teneur en uranium. Au debut ces teneur etait mesuré  Les slabs~M0 sont dites stériles, car elle contient tellement peu d'uranium que l'on ne souhaite pas les traiter. Les classes~M1 et M2 subissent un traitement que l'on dit statique, car c'est slab sont empiler et l’on attend que l'uranium descend par gravité jusqu'un bas. Enfin les slabs de classe supérieure reçoivent un traitement dynamique où en fonction de leur classe elles seront dissoutes avec plus ou moins d'acide selon leurs classes. Il est donc important de bien classer les slabs, car sinon, soit on gaspille  de l'acide ou alors il reste des l'uranium non extrait dans notre refus
\par Avant, pour classer une slab, un Aide Prospecteur (AP) utiliser un compteur Geiger Müller en se penchant pour obtenir des mesures a plusieurs points sur le slab. Il était donc pénible de se pencher en permanence et donc en 2018 a été lancer le projet CanOp pour réduire la pénibilité de la tache.