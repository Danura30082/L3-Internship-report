\section{Amélioration de la CanOp}

Un des problèmes majeurs de la CanOp reste son poids relativement conséquent de 5,5~kg. Ce poids peut paraître léger, mais les opérateurs doive porter les sondes à bout de bras pendant un shift de 8h sous le soleil avec une température qui monte régulièrement au-dessus de 40~°C. Déjà lors de sa conception, on avez envisagée de changer l'armature d'aluminium pour de la fibre de carbone. 

Une grosse partie de mon travail a donc était d'étudier et de proposer des solutions à ce problème. Assez rapidement, trois avenues d'amélioration sont apparues.
\begin{enumerate}
    \item Alléger le GPS 
    \item alléger les sondes
    \item repartir l'effort sur l'opérateur
\end{enumerate}

\subsection{Alléger le GPS}
Actuellement, le GPS est une pièce monolithique fournie par Ophelia (voir \cref{ssec_Gps_differenciel}) qui calcule en interne la position corrigée de la sonde. Une solution pourrait être de fracturer les différentes parties du GPS et de délocaliser le calcul de la position et de sa correction a appliqué depuis la tablette de l'opérateur en laissant l'antenne sur la sonde. D'autres solutions à partir de puce intégrées pourraient également mener à des économies de poids.

\subsection{Alléger les sondes}
Les sondes sur les CanOp sont des sondes en deux pièces composées d'un cristal scintillateur et d'un détecteur (ici un tube photomultiplicateur) (voir \cref{ssec_sonde}). Ces sondes sont relativement lourdes et ne sont pas solidaires de ce qui pose des problèmes de déconnections accidentel et d'infiltration de poussière/d'eau. %fuite de lumiere
J'ai donc cherché des sondes qui pourraient être plus étanches et/ou plus légères. En faisant c'est rechercher je suis tombé par accident sur des capteurs solides state SiPM qui pourrait remplacer les tubes photomultiplicateur. Ces composants sont devenus un remplacement viable de PMT que très récemment et n'était donc pas disponibles pour la première version de la CanOp. Ces composants présents de nombreux avantage~:
\begin{itemize}
    \item plus léger
    \item peu cher a fabriquer en masse (capitalisation sur les avancées faite en lithographie)
    \item basse tension (5~V vs 1000-2000~V pour les PMT) $\rightarrow$ simplification des électroniques
    \item Plus robuste (voir la fêlure sur la \cref{fig_PMT})
\end{itemize}

Ces avantages permettent de produire des sondes gamma pesant 25~g\cite{} pour les plus petits comparer à environ 150~g\cite{} pour les sondes classiques. De plus, ces sondes sont bien plus faciles à rendre étanche, car il n'y a plus besoin de séparer l'électronique du cristal.
